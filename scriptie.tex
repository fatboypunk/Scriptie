%%%%%%%%%%%%%%%%%%%%%%%%%%%%%%%%%%%%%%%%%
% Afstudeer scriptie
%%%%%%%%%%%%%%%%%%%%%%%%%%%%%%%%%%%%%%%%%

%----------------------------------------------------------------------------------------
%	PACKAGES AND OTHER DOCUMENT CONFIGURATIONS
%----------------------------------------------------------------------------------------

\documentclass[a4paper,12pt,oneside]{report} 
% Default font size is 12pt, it can be changed here

\usepackage[utf8]{inputenc}
\usepackage[T1]{fontenc}
\usepackage{palatino}


\usepackage{blindtext}
\usepackage[ampersand]{easylist}
\usepackage{fancyhdr}

\usepackage[xindy,toc]{glossaries}
\usepackage[font={small,it}]{caption}
\usepackage{tocloft}
%\usepackage[options]{natbib}
\usepackage[numbers]{natbib}

\usepackage[table,usenames,dvipsnames]{xcolor}
\usepackage{tabularx}

\usepackage{hyperref}
\hypersetup{
    colorlinks,
    citecolor=black,
    filecolor=black,
    linkcolor=black,
    urlcolor=black
}

\usepackage[dutch]{babel}
\selectlanguage{dutch}

\usepackage[left=1.7in]{geometry} % Required to change the page size to A4
\geometry{a4paper} % Set the page size to be A4 as opposed to the default US Letter

\usepackage{graphicx} % Required for including pictures

\usepackage{float} % Allows putting an [H] in \begin{figure} to specify the exact location of the figure
\usepackage{wrapfig} % Allows in-line images such as the example fish picture

\usepackage{lipsum} % Used for inserting dummy 'Lorem ipsum' text into the template

\linespread{1.2} % Line spacing

\usepackage{array,ragged2e,pst-node,pst-dbicons}

\graphicspath{{./img/}} % Specifies the directory where pictures are stored


\usepackage{listings}
\usepackage{textcomp}

\definecolor{javared}{rgb}{0.6,0,0} % for strings
\definecolor{javagreen}{rgb}{0.25,0.5,0.35} % comments
\definecolor{javapurple}{rgb}{0.5,0,0.35} % keywords
\definecolor{javadocblue}{rgb}{0.25,0.35,0.75} % javadoc
\definecolor{lbcolor}{rgb}{0.9,0.9,0.9} % javadoc
\definecolor{darkblue}{rgb}{0.0,0.0,0.6}
\definecolor{cyan}{rgb}{0.0,0.6,0.6}
\definecolor{Grey}{gray}{0.8}

\lstset{
	language=Java,
	basicstyle=\scriptsize,
	keywordstyle=\color{javapurple}\bfseries,
	stringstyle=\color{javared},
	commentstyle=\color{javagreen},
	morecomment=[s][\color{javadocblue}]{/**}{*/},
	backgroundcolor=\color{lbcolor},
	numbers=left,
	numberstyle=\color{black},
	stepnumber=1,
	numbersep=10pt,
	tabsize=2,
	showspaces=false,
	showstringspaces=false
	breaklines=true
}



\usepackage[compact]{titlesec}
\usepackage{pdfpages}

\setlength\aftertitleunit{\baselineskip}



\fancypagestyle{plain}{}



\newlength\titleindent
\setlength\titleindent{2cm}

\newcolumntype{g}{>{\columncolor{Grey}}c}

\pretocmd{\paragraph}{\stepcounter{subsection}}{}{}
\pretocmd{\subparagraph}{\stepcounter{subsubsection}}{}{}

\titleformat{\chapter}[block]
  {\normalfont\huge\bfseries}{\llap{\parbox{\titleindent}{\thechapter\hfill}}}{0pt}{}

\titleformat{\section}
  {\normalfont\Large\bfseries}{\llap{\parbox{\titleindent}{\thesection\hfill}}}{0em}{}

\titleformat{\subsection}
  {\normalfont\large}{\llap{\parbox{\titleindent}{}}}{0em}{\bfseries}

\titleformat{\subsubsection}
  {\normalfont\normalsize}{\llap{\parbox{\titleindent}{\thesubsubsection}}}{0em}{\bfseries}

\titleformat{\paragraph}[runin]
  {\normalfont\large}{\llap{\parbox{\titleindent}{\thesubsection\hfill}}}{0em}{}

\titleformat{\subparagraph}[runin]
  {\normalfont\normalsize}{\llap{\parbox{\titleindent}{\thesubsubsection\hfill}}}{0em}{}

\titlespacing*{\chapter}{0pt}{0pt}{20pt}
\titlespacing*{\subsubsection}{0pt}{3.25ex plus 1ex minus .2ex}{1.5ex plus .2ex}
\titlespacing*{\paragraph}{0pt}{3.25ex plus 1ex minus .2ex}{0em}
\titlespacing*{\subparagraph}{0pt}{3.25ex plus 1ex minus .2ex}{0em}

\setcounter{tocdepth}{1}
\cftsetindents{chapter}{0in}{0.5in}
\cftsetindents{section}{0in}{0.5in}

\newcolumntype{C}[1]{>{\Centering}p{#1}}
\def\Tab#1{\tabular{C{3cm}}\rule[-5mm]{0pt}{1cm}#1\\\hline
                           ~\\\hline~\endtabular}
\seticonparams{entity}{shadow,fillcolor=black!20,fillstyle=solid,framesep=0pt}

\input{glosseries}

\begin{document}

%----------------------------------------------------------------------------------------
%	Heading
%----------------------------------------------------------------------------------------

\pagestyle{fancy}
\renewcommand{\headrulewidth}{0.0pt}
\headheight = 54pt
 
\lhead{\includegraphics[width=0.4\linewidth]{nidaros-logo.png}}
\rhead{\includegraphics[width=0.4\linewidth]{hanze_logo.png}}






%----------------------------------------------------------------------------------------
%	TITLE PAGE
%----------------------------------------------------------------------------------------

\begin{titlepage}
\oddsidemargin 1cm

\newcommand{\HRule}{\rule{\linewidth}{0.5mm}} % Dopenrightefines a new command for the horizontal lines, change thickness here

\center % Center everything on the page

\textsc{\LARGE Hanzehogeschool}\\[1.0cm] % Name of your university/college

\textsc{\large \textit{Informatica} }\\[0.5cm] % Major heading such as course name


\HRule \\[0.4cm]
{ \huge \bfseries Afstudeer scriptie}\\[0.4cm] % Title
\HRule \\[6cm]



\includegraphics[width=\linewidth]{nidaros-logo.png}\\
\today, Hoogeveen
\end{titlepage}

\large
\emph{Auteur:}\\
Marcel Horlings\\
 \textsc{351254} \\
Opleiding: Informatica \\



\emph{Stagedocent:}\\
Jacob Mulder\\
Hanzehogeschool Groningen \\



\emph{Stagebedrijf:}\\
  Nidaros\\
	Pascal Hakkers\\
	Stoekeplein 1a\\
	7902 HM, Hoogeveen


\pagenumbering{gobble}

%----------------------------------------------------------------------------------------
%	Woordvooraf
%----------------------------------------------------------------------------------------
\chapter*{Woord vooraf}
\lipsum[1]

%----------------------------------------------------------------------------------------
%	Samenvatting
%----------------------------------------------------------------------------------------
\chapter*{Samenvatting}
\lipsum[1]





%----------------------------------------------------------------------------------------
%	Index
%----------------------------------------------------------------------------------------
\newpage
\renewcommand*\contentsname{Inhoud}
\tableofcontents
\cleardoublepage 
\pagenumbering{arabic}




%----------------------------------------------------------------------------------------
%	Inleiding
%----------------------------------------------------------------------------------------

\chapter{Inleiding} 
Inleiding

%----------------------------------------------------------------------------------------
%	Nidaros
%----------------------------------------------------------------------------------------

\section{Nidaros} 
Nidaros is een bedrijf dat adviseert en ondersteuning biedt bij IT-gerelateerde bedrijfsprocessen. Het doel van Nidaros is het stimuleren van bedrijfsprocessen, het informeren van procesverantwoordelijkheden en het bewust worden van wat IT kan toevoegen aan bedrijven en organisaties. Nidaros is een klein bedrijf met ongeveer twaalf werknemers, die op veel markten in de IT bezig is. Het biedt advies op het gebied van softwareontwikkeling en geven advies op basis van testmanagement en op basis van een informatieanalyse. Nidaros maakt zelf websites en software voor klanten en voeren optimalisaties door in bestaande websites. Daarnaast doet ze ook een stuk ICT-beheer binnen bedrijven, en voeren ze reparaties van computers en iPhones uit. 

\subsection{Consultancy}
In de Consultancy tak van Nidaros worden werknemers gedetacheerd naar andere bedrijven om daar te helpen met het inbrengen van externe systemen, en voor het testen van systemen.
% \lipsum[1]
\subsection{Solutions}
% \lipsum[1]
Bij Solutions worden producten verkocht aan gebruikers. Deze producten kunnen ingekocht zijn, zelf gemaakt zijn of een combinatie hiervan. Daarnaast levert Solutions ook diensten op het gebied van ICT.

\subsection{Organisatiestructuur}
% \lipsum[2]
Nidaros is een klein bedrijf met 12 werknemers in dienst.

\subsection{De primaire processen}
\lipsum[3]


\section{Probleemstelling}
Bij elke aankoop die gedaan wordt in een winkel wordt er een bonnetje meegegeven. Met dit bonnetje kan de garantie van een product verhaald worden wanneer het product het begeeft. Voor veel mensen is het bijhouden van alle bonnetjes die bij elk apparaat verkregen wordt en lastige taak. Ze vergeten wanneer hoe lang het bonnetje nog geldig is of wanneer de garantie afloopt en veel bonnetjes raken ook kwijt.
Voor de mensen die hier last van hebben wordt de ScanjeBon app ontwikkeld. Met deze app kunnen de bonnetjes gemakkelijk via de smartphone opgeslagen worden. Waardoor gebruikers de bonnetjes altijd op één centrale plek hebben. Gebruikers van de app zullen ook genotificeerd worden wanneer bonnen aflopen en ze kunnen de bonnetjes overzichtelijk uit elkaar houden.

\section{Inhoud van dit rapport}
\lipsum[5]

\chapter{Plan van aanpak}
De looptijd van het project bedraagt twintig weken. Deze twintig weken zal verdeeld worden over vier fases zoals beschreven in RUP. De eerste fase is Inception, in deze fase wordt er voor gezorgd dat iedereen betreffende het project hetzelfde beeld krijgt over het project. De tweede fase, Elaboration genaamd, wordt gebruikt als de eerste iteratie. 

Zo wordt er een werkend product opgeleverd, maar worden ook dingen gedaan als het opzetten van de ontwikkelomgeving en het versiebeheersysteem. De derde fase, Construction, is waar het ontwikkelen gebeurd. Hier zal het overgrote deel van de Scanjebon app gemaakt worden. Dit wordt gedaan in iteraties en na elke iteratie wordt de app getoond aan de deelnemende stakeholders. 

De laatste fase is de Transition, de Scanjebon app is hier klaar voor gebruik en kan in de markt gezet worden. Daarnaast zal hier de overdracht plaats vinden van de broncode, de documentatie en de ontwikkelomgeving. De fases zullen er samen voor zorgen dat er een Scanjebon app voor minimaal één platform gemaakt zal worden met daarnaast een ontwikkelomgeving waar een volgende ontwikkelaar mee verder kan. De planning staat als gantt chart in \ref{chap:planning}.
\begin{description}
Overdracht app en broncode.Overdracht app en broncode.
  \item[Fase 1: Inception] 
    \item \mbox{}
    \begin{itemize}
      \item Projectgoedkeuring.
      \item Onderzoek talen / platform.
      \item Onderzoek ontwikkelomgeving.
    \end{itemize}
  \item[Fase 2: Elaboration]
    \item \mbox{}
    \begin{itemize}
      \item Opzet versiebeheersysteem..
      \item Wireframes maken
      \item Bouwen app:
      \item Foto’s maken / laden uit galerij
    \end{itemize}
  \item[Fase 3: Construction]
    \item \mbox{}
    Bouwen app:
    \begin{itemize}
      \item Gegevens invullen en meesturen
      \item Bonnen bekijken in een lijst en apart 
      \item Gegevens automatisch uit de foto halen d.m.v. OCR
      \item Ontwikkeling REST API:
      \item Opzetten framework
      \item REST infrastructuur opzetten.
    \end{itemize}
  \item[Fase 4: Transition] 
    \item \mbox{}
    \begin{itemize}
      \item Advies taal platform
      \item Ontwikkelomgeving / ontwikkelstraat
      \item Overdracht app en broncode.
    \end{itemize}
\end{description}

%----------------------------------------------------------------------------------------
%	Keuze platform
%----------------------------------------------------------------------------------------

\chapter{Keuze platform}
\lipsum[8]
\section{Criteria}
\lipsum[9]

\section{Uiteindelijke keuze}
\lipsum[10]



%----------------------------------------------------------------------------------------
%	CONCLUSION
%----------------------------------------------------------------------------------------

\chapter{Conclusie}
\lipsum[11-14]




\appendix
%----------------------------------------------------------------------------------------
%	Persoonlijke ontwikkeling
%----------------------------------------------------------------------------------------
\chapter{Persoonlijke ontwikkeling}
\lipsum[1]

%----------------------------------------------------------------------------------------
% Persoonlijke ontwikkeling
%----------------------------------------------------------------------------------------
\chapter{Planning}
  \label{chap:planning}


%----------------------------------------------------------------------------------------
%	woordenlijst en bronnen
%----------------------------------------------------------------------------------------
\newpage

\printglossary


\renewcommand{\bibname}{Bronvermeldingen}

\bibliographystyle{plain}
\bibliography{bron}
\nocite{*}

%----------------------------------------------------------------------------------------



\end{document}
